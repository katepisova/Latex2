\documentclass[11pt,a4paper]{article}
\usepackage[left=30mm, right=15mm, top=20mm, bottom=20mm]{geometry}
\pagestyle{empty}
\usepackage{titlesec}
\renewcommand{\thesubsection}{\arabic{section}.\arabic{subsection}.}
\renewcommand{\thesection}{\arabic{section}.}
\usepackage[utf8]{inputenc}
\usepackage{indentfirst}
\setlength{\parindent}{1.2cm}
\setlength{\parskip}{0.2cm}
\usepackage[T2A]{fontenc}
\usepackage[english,russian, ukrainian]{babel}
\usepackage{enumitem}
\newcounter{defnumber}  % задаём имя счёчика 
\setcounter{defnumber}{4}  % устанавливаем его первое значение

\setlist{nolistsep, itemsep=6pt, leftmargin=0mm, itemindent=14mm}
\begin{document}
$$\bar{\Delta }=\frac{y_{n}-y_{1}}{n-1}=\frac{208.8-201.8}{8-1}=1.$$
\par Далі перевіримо нерівність: $\sigma_{\textup{\textit{зал}}}^2\leq \rho ^{2}$.
\\ Маємо, $\sigma_{\textup{\textit{зал}}}^2\frac{\sum\limits_{t=1}^{n}\left ( y_{t}-\bar{y_{\Delta }} \right )^{2}}{n}=\frac{1.85}{8}=0.231$,
$\rho ^{2}=\frac{1}{2}\sum\limits_{t=1}^{n}\Delta t^{2}=\frac{1}{2}\cdot \frac{7.42}{8}=0.464$.
\par Нерівність $\sigma_{\textup{\textit{зал}}}^2 < \rho ^{2}$ і тому можна робити прогноз:
$$\hat y_{\textup{\textit{вер.}}} = 208.8 + 1 \cdot 1=209.8 \:\textit{квт}/\textit{год}, $$
$$\hat y_{\textup{\textit{жов.}}} = 208.8 + 1 \cdot 2=210.8 \:\textit{квт}/\textit{год}. $$
\begin{center}
\textbf{4. Прогнозування методом середнього темпу роста}
\end{center}
Цим методом роблять прогноз у випадку, якщо тенденція характеризується експоненціальною кривою. Прогноз методом середнього темпу роста робиться за формулою:
$$ \hat y_{t+L}=y_{t}\times \bar{T_{p}^{L}},$$
$\bar{T_{p}} =\sqrt[n-1]{\frac{y_{n}}{y_{1}}}$ - середній темп росту.
\par Зазначимо, що емпірічних значень початкового ряду повинна дорівнювати сумі значень ряду, вирівнених за середнім темпом росту.Якщо ці суми не співпадають, то це означає:
\begin{enumerate}
\item початковий ряд має не експоненційну тенденцію,
\item випадкові фактори мають великий вплив на формування ряду.
\end{enumerate}
\par \textbf{Приклад.} По даним споживання електроенергії в місті розрахувати прогноз на червень і липень методом середнього темпу роста. 
\\
\\
\vspace{\baselineskip}
\begin{tabular}[central]{ | l | l | l | }
\hline

\hline
\end{tabular}
\par Середній темп росту такий:
$$\bar{T_{p}}=\sqrt[n-1]{\frac{y_{n}}{y_{1}}}=\sqrt[5-1]{\frac{17}{10}}=1.14.$$
\par Тоді прогноз споживання електроенергії у місті такий:
$$\hat y_{\textup{\textit{червень}}}=17\cdot 1.14^{1}=19.4\:\textit{млн.квт}/\textit{год},$$
$$\hat y_{\textup{\textit{червень}}}=17\cdot 1.14^{2}=22.1\:\textit{млн.квт}/\textit{год}.$$
\par Розглянуті методи прогнозування є найпростішими і прогнози, зроблені за ними, не дуже точні. Тому ці методи використовуються тільки для короткострокових прогнозів.
\newpage
\section*{Прогнозування рядів динаміки, що не мають тенденції}
При ровз'язуванні соціально-економічних задач інколи зустрічаються часові ряди без тендеції. Особливість прогнозування в таких рядах полягає в тому, що використовувати методи статистичного прогнозування для отримання точного або інтервального прогнозу недоцільно. У цьому випадку доцільно застосовувати ймовірностні статистичні методи прогнозного оцінювання.
\par Ймовірностні методи оцінювання не дають можливості зробити точкову кількісну характеристику прогнозу. Вони дають оцінку ймовірності того, що значенння прогнозу буде більше або менше значення останнього рівня початкового ряду. Ймовірностні методи прогнозування дають менш точні прогнозні оцінки і мають невизначеність.
\par На практиці, при аналізі часових рядів соціально-економічних явищ, що не мають тенденції, найбільше розповсюдження серед ймовірностних методів прогнозування, отримав метод, який ґрунтується на використанні закону розподілу Пуассона з густиною: $\rho = e^{\alpha }.$

\end{document}
 
